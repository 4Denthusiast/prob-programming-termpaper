\documentclass{article}

%% BEGIN {Luke's Macros}
%%
\newif\ifdraft
\drafttrue %% To hide all comments and highlighting, just comment this line out 

\ifdraft
\usepackage[draft]{commenting}
\else
\usepackage[nompar]{commenting}
\fi

\declareauthor{lo}{L}{red}
\authorcommand{lo}{comment}

\declareauthor{akr}{A}{blue}
\authorcommand{akr}{comment}
\setdefaultauthor{akr}

\makeatletter
\renewcommand{\comm@todo@mpar}[1]{}
\makeatother

\def\divider{%
  \leavevmode\leaders\hrule height 0.6ex depth \dimexpr0.4pt-0.6ex\hfill%
  \kern0pt%
}
\newcommand{\ONGOING}[1]{%
  \draftpar\comment*{\divider\MakeUppercase{Ongoing~#1}\divider}\draftpar%
}

\renewcommand{\OpenCommentBraket}{$\lceil$}
\renewcommand{\ClosedCommentBraket}{$\rfloor$}

\usepackage[square]{natbib}
\setcitestyle{aysep={}}

\definecolor{oxblue}{RGB}{0,33,71}
\definecolor{oxgold}{HTML}{a0630a}
\usepackage[final,
bookmarks,
bookmarksopen,
colorlinks,
final,
linkcolor=oxred,
citecolor=oxgold,
pdfstartview=FitH ]{hyperref}
%%
%% END {Luke's Macros}

\usepackage{amsmath}
\usepackage{amsfonts}
\usepackage{verbatim}
\newcommand{\Y}{\textsf{Y}}
\newcommand{\sample}{\textsf{sample}}
\DeclareMathOperator{\red}{red}

\begin{document}

\lo{\LaTeX. For references, I recommend using the package {\tt natbib} with {\tt apalike} bibliography style (author-year is more reader-friendly), and placing the references in a separate {\tt .bib} file.
I simply copy-and-paste the bib data from DBLP e.g.~\url{https://dblp.uni-trier.de/pers/hd/o/Ong:C==H=_Luke}.}

\changed[akr]{[Email 22 Dec: I wrote a proof of termination based on ranking supermartingales (a slight strengthening of what you suggested), which is attached (I haven't made a Git repo yet because I don't know where you want it hosted).

I thought some more about termination in the case of a language with a score function. 
It isn't even clear what termination means in that case.
None of the papers I've found include both unbounded scoring and a definition of termination.
\lo{Can you explain your concern? I don't see any problem. See, e.g., the operational semantics in \citep{MakOP20b}; see also \citep{MakOP20a}.} 
You said that the assumption of a.s. termination was often used. 
Do you have any examples where it's used in a context with unbounded scoring, as going with whatever definition that uses is probably most useful?
\lo{See \citep[Sec.~4, pp.~8 \& 9]{Wagner20} -- I am not sure if this addresses your question.}

The case of bounded scoring seems simpler definitionally at least, but still seems like a bad fit for supermartingales, as the martingale property is local in a way scoring isn't. 
I still need to think more about to what extent it can be made to make sense, though.]}

Following your suggestion, I will be providing a criterion for termination of programs in PPCF \citep{DBLP:journals/pacmpl/EhrhardPT18} based on ranking supermartingales. 
As it's more convenient for this proof, a sampling-based semantics \lo{You need to provide reference(s) for sampling-based semantics.} will be used instead of the original distributional semantics. 
I assume some roughly applicable equivalence is proven somewhere, but it doesn't seem that hard anyway.
\lo{Why not provide a proof as an appendix?}

\section{Ranking functions}
\lo{You should provide the definition of the PPCF: syntax and typing rules.}

Given a probabilistic program (i.e.~a term $M$), in order to construct a supermartingale to prove its a.s.~termination, 
\lo{\LaTeX. Place a space character right after ``.'' whenever the occurrence does not end a sentence (e.g.~{\tt a.s.\~{}termination} and {\tt i.e.\~{}a term}) so that \LaTeX\ does not allocate more white space than it should.}
a function to assign values to each reachable program state is necessary. Define a ranking function on $M$ to be a measurable function $f:Rch(M) \to \mathbb{R}$ such that
\begin{itemize}
    \item $f(N) > 0$ for all $N$ not in normal form
    \item $f(N) = 0$ for all $N$ in normal form
    \item $f(E[\Y N]) \geq 1 + f(E[N(\Y N)])$
    \item $f(E[\sample]) \geq \int_0^1 f(E[\underline{x}]) \, \mathrm{d}x$
    \item $f(E[R]) \geq f(E[N])$ for any other redex $R$, where $R \to N$
\end{itemize}

\section{Sampling semantics}
Let $ I = [0,1] \subset \mathbb{R} $, and let $S = I^{\mathbb{N}}$, with the $\sigma$-algebra and measure given by the limit of $1 \gets I \gets I^2 \gets \cdots$, where the maps are the projections that ignore the last element. 
The maps $\pi_h:S \to I, \; \pi_t:S \to S$ popping the first element are then measurable.

\lo{Is the induced $\sigma$-algebra the Borel algebra?
The basis elements of the topology of $S$ have the form $\big(\prod_{i \in I} U_i\big) \times \mathbb{R}^\omega$ where $I$ ranges over finite sets and each $U_i$ is open in $\mathbb{R}$.}

The one-step reduction is given by the function $\red : \Lambda \times S \to \Lambda \times S$ where
\begin{equation}
\red(M,s) = \left\{
    \begin{array}{ll}
        (E[N],s) & \text{if } M = E[R], R \to N \text{ and } R \neq \sample \\
        (\pi_h(s),\pi_t(s)) & \text{if } M =  \sample \\
        (M,s) & \text{if } M \text{ normal form}
    \end{array} \right .
\end{equation}
\lo{Your definition $\red$ is incomplete: it is missing the definition of $R \to N$.}

The limit $\red^\infty$ can then be defined as a partial function as $\lim_{n \to \infty} \red^n(M,s)$ whenever that sequence becomes constant by reaching a normal form.

%\begin{comment} 
\akr{\section{Recursive steps}
[I'm not sure whether this section will be needed any more.]

Define a reduction relation on PPCF terms that excludes the $\Y$ combinator (``nonrecursive reduction''), i.e. $E[M] \dashrightarrow E[N] \text{ if } M \to N$ for any redex $M$ not of the form $\Y M'$.

This reduction relation is strongly normalizing because of the simple type system, so for any term $M$ there is some bound $b$ such that nonrecursive reduction of $M$ must terminate in at most $b$ steps. Define $\red_{nr} : \Lambda \times S \to \Lambda \times S$ like $\red(M,s)$, but using nonrecursive reduction instead of reduction, and let $\red_Y(M,s) = \red^{b_M}_{nr}(M,s)$. The sequence $(r_\Lambda(M,s,n), r_S(M,s,n)) = r(M,s,n) = \left(\red_Y \circ \red \right)^n(\red_Y(M,s))$ is then a subsequence of $\left( \red^n(M,s) \right)_n$ consisting of only terms which are either normal form or of the form $E[\Y N]$ for some $N$.
}
%\end{comment}

\section{Supermartingales}
Given a term $M$ and a ranking function $f$ for it, define random variables on the probability space $S$ (where $s$ is a random variable) by
\begin{align*}
(M_n,s_n) & = \red^n(M,s) \\
y_0 & = 0 \\
%y_{n+1} & = \text{min} \{ k>y_n | M_k \text{ normal form or of the form } E[\Y N] \} \\
y_{n+1} & = \min \{ k \mid k>y_n, M_k \text{ normal form or of the form } E[\Y N] \}\\
M'_n & = M_{y_n} \\
X_n & = f(M'_n)
\end{align*}
\lo{I have made a trivial change to the third clause above.}
and define a filtration $\mathcal{F}_n = \sigma(M_k, k \leq n)$ (i.e. all the samples used up to step $n$).

\lo{So all the random variables just defined are constant functions. E.g.~$M_n$ is a constant function from $S$ to the measurable space of PPCF terms.}

\lo{\LaTeX. Use {\tt mid} rather than ``{\tt |}'' as separator in set comprehension. Reason: {\tt mid} is a binary operator (and so \LaTeX\ adds white space to both side); ``{\tt |}'' is just a symbol.}

The expectation of $f(M_{n+1})$ given $\mathcal{F}_n$ is trivially less than or equal to $f(M_n)$ in the cases that $M_n \neq E[\sample]$, and in the case of $\sample$,
\begin{align*}
& \mathbb{E}[f(M_{n+1}) \mid \mathcal{F}_n] \\
= & \mathbb{E}[f(M_{n+1}) \mid M_n = E[\sample],\, \mathcal{F}_n] \\
= & \mathbb{E}[f(E[\pi_h(s_n)]) \mid \mathcal{F}_n] \\
= & \int_0^1 f(E[\underline x]) \, \mathrm{d} x \qquad & \text{as }s_n\text{ is independent of } \mathcal{F}_n \\
\leq & f(E[\sample]) \qquad & \text{by assumption on } f \\
= & f(M_n),
\end{align*}
therefore the values of the ranking function $f(M_n)$ are a supermartingale with respect to $\mathcal{F}_n$.

Given $M'_n$, there is some finite bound on the number of reduction steps that can take place from $M'_n$ without a $\Y$-reduction step, because of the type system, therefore $y_{n+1}$ is (conditional on $\mathcal{F}_{y_n+1}$) a bounded stopping time, therefore $\mathbb{E}[f(M_{y_{n+1}}) \mid \mathcal{F}_{y_n+1}] \leq f(M_{y_n+1})$. If $f(M_{y_n}) > 0$, then $M_{y_n} = E[\Y N]$ for some $E, N$, therefore $M_{y_n+1} = N (\Y N)$ and $f(M_{y_n+1}) \leq f(M_{y_n}) - 1$, therefore if $X_n > 0$, $\mathbb{E}[X_{n+1} \mid \mathcal{F}_{y_n+1}] \leq X_n - 1$.

The expectation of $X_n$ therefore must tend to 0 as $n \to \infty$, therefore $X_n$ almost surely is less than or equal to 1 eventually, at which point $M'_{n+1}$ must be normal form, therefore $M$ terminates almost surely.


\lo{You should frame the result you have just proved, precisely and formally, using a \LaTeX\ {\tt lemma} or {\tt theorem} environment.}


\lo{\textbf{Further directions}

\smallskip

An obvious next step is to extend the result to the $\mathsf{score}$ construct.

The following are highly topical, and could form the basis of an interesting and novel DPhil thesis.
\begin{enumerate}
\item Devise methods for proving (positive) a.s.~termination. 

- For example, develop a type system satisfying the property: if a term is typable then it is (positively) a.s.~terminating.
See~\cite{DBLP:conf/ppdp/BreuvartL18,DBLP:conf/esop/LagoG17}.

- Another approach is to develop algorithms that synthesise ranking supermartingales, following, for example, \cite{DBLP:journals/pacmpl/AgrawalC018}.

\item Develop principles (e.g.~in the form of ``proof rules'') for reasoning about (positively) a.s.~termination, in the style of \cite{DBLP:journals/pacmpl/McIverMKK18}.


\item Design algorithms that synthesise probabilistic invariants (\emph{qua} martingales), \`a la \cite{SchreuderO19}; see also \cite{DBLP:journals/pacmpl/HarkKGK20}.

\end{enumerate}}

\bibliographystyle{apalike}
\bibliography{references}

\iffalse
\begin{thebibliography}{9}
\bibitem{ppcf} Thomas Ehrhard, Michele Pagani, and Christine Tasson. Measurable cones and stable, measurable functions: a model for probabilistic higher-order programming. \emph{PACMPL}, 2(POPL):59:1–59:28, 2018. doi: 10.1145/3158147. URL \href{https://doi.org/10.1145/3158147}{https://doi.org/10.1145/3158147}.
\end{thebibliography}
\fi

\end{document}
