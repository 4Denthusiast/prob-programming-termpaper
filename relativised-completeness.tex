% !TEX root = ./probabilisticProgrammingMartingales.tex

\section{Relativised completeness}
\label{sec:Relativised completeness}

If every term reachable from a closed SPCF program is AST then the program is also AST, but not conversely 
(because a diverging term may be reachable with probability 0 from an AST term).

We aim to prove the following conjecture:

\begin{quote}
\emph{Given a closed SPCF term $M$, if every term reachable from $M$ is AST then $M$ admits an antitone ranking function.}
\end{quote}

\begin{enumerate}

\item Given an a.s.~finite stopping time $T$ (meaning: $T < \infty$ a.s.) adapted to a filtration, there exist supermartingale $(Y_n)_{n \in \omega}$ adapted to the same filtration and antitone function $\epsilon$ such that $(Y_n)_{n \in \omega}$ is an antitone strict supermartingale w.r.t.~$(\epsilon, T)$.
\lo{Is this what you had in mind?}

\item Let $(Y_n')_{n \in \omega}$ be an antitone strict supermartingale w.r.t.~$(\epsilon, T)$. 
Then $M$ admits an antitone ranking function.
\end{enumerate}

For 2, note that for every $N \in \mathit{Rch}(M)$, there exists $n \geq 0$ and $s \in S$ such that $\red^n(M, s) = (N, -)$.
Define a function $f : \mathit{Rch}(M) \to \nnReal$ by $f(N) := Y_n'(s)$.

For 1, consider the following:
\begin{proof}
If $T$ is bounded by $b$ a.s.~(for $n$ constant), the statement is trivially true by taking $\epsilon$ to be constantly 1, and $Y_n = b - n$.
Otherwise, let $(T_n)_{n \in \omega}$ be defined recursively such that $T_0 = 0$, $T_{n+1} > T_n$ and $\mathbb P[T < T_{n+1}] \leq \frac 1 4 \mathbb P[T < T_n]$. We will then define a supermartingale $(Y_n)_{n \in \omega}$ and nonrandom sequence $(Y'_n)_{n \in \omega}$ such that $Y_n = 0$ iff $T < n$, $Y_n = Y'_n$ iff $T \geq n$, and $Y'_n \geq 2^k$ for $n \geq T_k$.

$\epsilon$ is defined piecewise and recursively in such a way as to force all the necessary constraints to hold:
\begin{align*}
    \text{for }x \in [0,1),\ &\epsilon(x) = 1 \\
    \text{for }x \in [2^k,2^{k+1}),\ &\epsilon(x) = \max \left (\epsilon(2^{k-1}), \frac{2^k}{T_{k+1}-T_k} \right ).
\end{align*}
This ensures that
\begin{itemize}
\item $\epsilon$ is weakly decreasing.
\item $\epsilon$ is bounded below by 0.
\item For $x \geq 2^k$, $\epsilon(x) \leq \frac{2^k}{T_{k+1}-T_k}$.
\end{itemize}

$Y'_n$ is then defined recursively by
\begin{align*}
    Y'_0 = 2,
    Y'_{n+1} = (Y'_n - \epsilon(Y'_n)) \frac{\mathbb P[T > n]}{\mathbb P[T > n+1]}.
\end{align*}

The fact $Y'_{T_k} \geq 2^{k+1}$ is proven by induction on $k$. For base case, $2 \geq 2$, as required. For the inductive case $k+1$, we first do another induction to prove that $Y'_n \geq 2^k$ for all $T_k \leq n \leq T_{k+1}$. The base case of this inner induction follows from the outer induction hypothesis a fortiori. take the greatest $k$ such that $n > T_k$. By the induction hypothesis, $Y'_{T_k} \geq 2^k$.
\begin{align*}
    Y'_n & = Y'_{T_k} \frac{\mathbb P[T > T_k]}{\mathbb P[T > n]} - \sum_{m=T_k}^{n-1} \epsilon(Y'_m) \frac{\mathbb P[T > m+1]}{\mathbb P[T > n]} \\
    & \geq \frac{\mathbb P[T > T_k]}{\mathbb P[T > n]} (Y'_{T_k} - \sum_{m=T_k}^{n-1} \epsilon(Y'_m)) \\
    & \geq \frac{\mathbb P[T > T_k]}{\mathbb P[T > n]} (Y'_{T_k} - \sum_{m=T_k}^{n-1} \frac{2^k}{T_{k+1}-T_k}) \\
    & = \frac{\mathbb P[T > T_k]}{\mathbb P[T > n]} (Y'_{T_k} - \frac{2^k (n - T_k)}{T_{k+1}-T_k}) \\
    & \geq \frac{\mathbb P[T > T_k]}{\mathbb P[T > n]} (Y'_{T_k} - 2^k) \\
    & \geq \frac{\mathbb P[T > T_k]}{\mathbb P[T > n]} (2^{k+1} - 2^k) \\
    & = \frac{\mathbb P[T > T_k]}{\mathbb P[T > n]} 2^k \\
    & \geq 2^k
\end{align*}
as required. Substituting $n = T_{k+1}$ and using the fact that $\frac{\mathbb P[T > T_k]}{\mathbb P[T > T_{k+1}]}$, the same reasoning gives $Y'_{T_{k+1}} \geq 2^{k+2}$, as required.

Although the stronger induction hypothesis was necessary for the induction to work, the only reason this is needed is to prove that $Y'_n > 0$ for all n, which implies that $Y_n \geq 0$. The other condition for $(Y_n)$ to be an antitone-strict supermartingale is \akr{The inference from the third to the fourth line is actually incorrect if $\mathcal F$ is not coarsest filtration to which $T$ is adapted. This error may require a substantial change to the proof to correct, but I thought you'd want to see it anyway rather than waiting for me to fix it.} \\
\begin{align*}
    & \expect{Y_{n+1} \mid \mathcal F_n} \\
    = & \expect{Y'_{n+1} 1_{T \geq n+1} \mid \mathcal F_n} \\
    = & Y'_{n+1} \expect{1_{T \geq n+1} \mid \mathcal F_n} \\ 
    = & Y'_{n+1} 1_{T \geq n} \frac{\mathbb P[T > n+1]}{\mathbb P[T > n]} \\
    = & (Y'_n - \epsilon(Y'_n)) 1_{T \geq n} \\
    = & Y_n - \epsilon(Y'_n) 1_{T \geq n}
\end{align*}
therefore $(Y_n)_n$ is an antitone strict supermartingale with respect to $(T,\epsilon)$.
\end{proof}
\lo{Ongoing.}


