% !TEX root=./main.tex

\rankPast*
\begin{proof}%\rankPast
We first prove (by induction on $n$): for all $n \geq 0$
\[
%\forall n \geq 0 \, . \, 
\expect{Y_n} \leq \expect{Y_0} - \epsilon \cdot \big(\textstyle\sum_{i=0}^{n-1} \mathbb{P}[T > i]\big).
\]
It follows that $\sum_{i=0}^{\infty} \mathbb{P}[T > i]$ converges; hence we have $\lim_{i \to \infty} \mathbb{P}[T > i] = 0$ and so $\mathbb{P}[T < \infty] = 1$.
It then remains to observe: if $T < \infty$ a.s., then $\expect{T} = \sum^\infty_{i=0} \mathbb{P}[T > i]$.
\end{proof}

\rankableAndStrictRankable*
We shall obtain the theorem as a corollary of a technical lemma (\Cref{lem:key rankable}).

We say that a given PPCF term is: 
\emph{type 1} if it has the shape $E[\tY \lambda x. N]$, \emph{type 2} if it has the shape $E[\tsample]$; \emph{type 3} if it has the shape $E[R]$ where $R$ is any other redex, and \emph{type 4} if it is a value.
Henceforth fix an $n \geq 0$, and define $\mathbf{T}_i := \set{s \mid M_n(s) \hbox{ is type $i$}}$.
It is straightforward to see that each $\mathbf{T}_i \in \calF_n$, and $\set{\mathbf{T}_1, \mathbf{T}_2, \mathbf{T}_3, \mathbf{T}_4}$ is a partition of $\entrosp$.
Hence it suffices to prove the following lemma.

\begin{restatable}[Technical]{lemma}{TechnicalRankingSupermartingale}
\label{lem:key rankable}
For all $i \in \set{1, 2, 3, 4}$ and $A \in \calF_n$
\[
\int_A \mu_{\entrosp}(\dif s) \, f(M_{n+1})[s \in \mathbf{T}_i] \leq \int_A \mu_{\entrosp}(\dif s) \, f(M_n)[s \in \mathbf{T}_i]
\; \footnote{where (the Iverson bracket) $[P] := 1$ if the statement $P$ hold, and 0 otherwise.}
\] 
%$\int_A \mu_{\entrosp}(\dif s) \, f(M_{n+1}) \leq \int_A \mu_{\entrosp}(\dif s) \, f(M_n)$, i.e., 
Hence $\mathbb{E}[f(M_{n+1}) \mid \calF_n] \leq f(M_n)$ a.s.
\end{restatable}

First, some notation.

Given an $\mathbb N$-indexed sequence (e.g.~$s \in \entrosp$) and $m \geq 0$, we write $s_{\leq m} \in I^m$ to mean the prefix of $s$ of length $m$.
For $n \geq 0$, define
\begin{align*}
%M_n(s) &:= \pi_0 (\red^n(M, s))\\
\ndraw{n}{s} &:= |\{k < n \mid \exists E: M_k(s) = E[\tsample]\}|
\end{align*}
so that $\pi_1 (\red^n(M, s)) = %\overbrace{\pi_t( \cdots (\pi_t}^{\ndraw n s}(s))$. 
\pi_t( \cdots (\pi_t (s))$, with $\pi_t$ applied ${\ndraw n s}$ times.
%and the filtration $\mathcal{F}_n := \sigma(M_1, \cdots, M_n)$.
The $\calF_n$-measurability of $M_n$ (and hence of $\#\mathrm{draw}_n$) follows from \cite{DBLP:conf/icfp/BorgstromLGS16}.
%$\#\mathrm{draw}_n$ is a stopping time (bounded by $n$). 
\iffalse
\akr{$\#\mathrm{draw}_n$ is not a stopping time. It doesn't look like you actually use this claim anyway, so it should just be fine to remove, but did you mean something different?} \lo{I agree, and I don't actually use this claim.}
\fi
Take $s \in A \in \calF_n$ with $\ndraw{n}{s} = l$.
For any $s'\in \entrosp$, if $s_{\leq l} = s_{\leq l}'$ then $s' \in A$.
It follows that ${\set{s_{\leq l}} \cdot I^{\mathbb N}} \subseteq A$.
\lo{NOTE. To fix notation:
\(
\sigma(M_{n}) = \sigma\big(\set{M_{n}^{-1}(\alpha_j, U_{\alpha_j})
\mid \alpha_j\in \mathsf{Sk}_j, U_{\alpha_j} \in \mathcal{B}(\Real^j), j \geq 0}\big)
\)
}
%Take $M \in \Lambda^0$ (closed PPCF terms). Define, for each $n \in \omega$, the random variable $M_n : \entrosp \to \Lambda^0$ by $M_n(s) := \pi_0 (\red^n(M, s))$.

\begin{proof}
We show the non-trivial case of $i = 2$.
First we express 
\begin{equation}
f(M_{n+1})[s \in \mathbf{T}_2] = \sum_{i \in \calI} 
f(E_i[\sigma_i(s)][\underline{\rho(s)}])[s \in U_i]
\label{eqn:f 2 n+1}
\end{equation}
where 
\begin{itemize}
\item $\calI$ is a countable indexing set
\item $E_i[\cdot][\tsample] \in \mathsf{Sk}_{j_i}$, and $\sigma_i : \entrosp \to \Real^{j_i}$, and $\rho(s) := \pi_h(\pi_1(\red^n(M, s))) = \pi_h(\pi_t^{l_i}(s)) \in \Real$ 
\item $\set{U_i}_{i \in \calI}$ is a partition of $\mathbf{T}_2$, 
where each $U_i$ is determined by a skeletal environment $E_i$ and a number (of draws) $l_i \leq n$, so that $U_i$ is the set of traces where after $n$ reduction steps, $l_i$ samples have been used, and the term has reached $E_i[r][\tsample]$ for some $r \in \Real^{j_i}$.
%\(
%s \in U_i
%\iff
%(E_i[\underline{r}][{\tsample}], \pi_t^{l_i}(s)) = \red^n(M, s),
%\;
%\hbox{for some }r \in \Real^{j_i}.
%\)
(We use the (measurable) function $\sigma_i : \entrosp \to \Real^{j_i}$ to skolemise the existentially quantified $r$.)
Equivalently
\[
U_i := \#\mathrm{draw}_n^{-1}[l_i] \cap M_n^{-1}[\{E_i[r][\tsample] \mid r \in \Real^{j_i}\}] \in \calF_n.
\]
%\lo{I.e.~$(E_i[\underline{\sigma_i(s)}][{\tsample}], \pi_t^{l_i}(s)) = \red^n(M, s)$ iff $s \in U_i$.}
\end{itemize}

Observe that if $s \in U_i$ then $\set{s_{\leq l_i}} \cdot I^{\mathbb N} \subseteq U_i$;
in fact $(U_i)_{\leq l_i} \cdot I^{\mathbb N} = U_i$.
%Moreover, for any $A \in \calF_n$, if $s \in A$ then $\set{s_{\leq l}} \cdot I^{\mathbb N} \subseteq A$.
This means that for any measurable $g : \entrosp \to \Real_{\geq 0}$, if $g(s)$ only depends on the prefix of $s$ of length $(l_i+1)$, then, writing $\widehat{g} : I^{l_i+1} \to \Real_{\geq 0}$ where $g(s) = \widehat{g}(s_{\leq l_i+1})$, we have: for any $A \in \calF_n$ 
\begin{equation}
\int_{A \cap U_i}  \mu_{\entrosp}(\dif s) \, g(s) = 
\int_{(A \cap U_i)_{\leq l_i+1}} \Leb_{l_i+1}(\dif t) \, \widehat{g}(t)
\label{eqn:truncate}
\end{equation}
%(For simplicity, we will write $g' = g$ in the following.)
%where $A_{\leq m} := \set{s_{\leq m} \mid s \in A}$. 

Take $s \in U_i$, and set $l = l_i$.
Plainly $\sigma_i(s)$ depends on $s_{\leq l}$, and $\rho(s)$ depends on $s_{\leq l +1}$.
Take $u \in (U_i)_{\leq l}$.
It then follows from the definition of ranking function that
\[%\textstyle
\int_I \Leb(\dif r) \, f(E_i[\widehat{\sigma_i}(u)][\underline{r}])) \leq f(E_i[\widehat{\sigma_i}(u)][\tsample])).
\]
Take $A \in \calF_n$, and integrating both sides, we get
\begin{align*}
& \int_{(A \cap U_i)_{\leq l}} \Leb_l(\dif u) \, \int_I \textrm{Leb}(\dif r) \, f(E_i[\widehat{\sigma_i}(u)][\underline{r}])) \\
& \leq \int_{(A \cap U_i)_{\leq l}} \Leb_l(\dif u) \, f(E_i[\widehat{\sigma_i}(u)][\tsample])).
\end{align*}
Since $\Leb_{l+1}$ is the (unique) product measure satisfying $\Leb_{l+1}(V \times B) = \Leb_l(V) \cdot \Leb(B)$, and $(U_i)_{\leq l_i} \cdot I^{\mathbb N} = U_i$, we have
%it follows from the preceding observation that
\begin{align*}
& \int_{(A \cap U_i)_{\leq l+1}} \Leb_{l+1}(\dif u') \, f(E_i[\widehat{\sigma_i}(u')][\underline{\widehat{\rho}(u')}])) \\
& \leq 
\int_{(A \cap U_i)_{\leq l}} \Leb_l(\dif u') \, f(E_i[\widehat{\sigma_i}(u')][\tsample]))
\end{align*}
and so, by (\ref{eqn:truncate})
\begin{align}
& \int_{A \cap U_i} \mu_{\entrosp}(\dif s) \, f(E_i[\sigma_i(s)][\underline{\rho(s)}])) \nonumber \\
& \leq 
\int_{A \cap U_i} \mu_{\entrosp}(\dif s) \, f(E_i[\sigma_i(s)][\tsample])).
\label{eqn:a u ui}
\end{align}

Now, integrating both sides of (\ref{eqn:f 2 n+1}), we have
\begin{align*}
& \int_A \mu_{\entrosp}(\dif s) \, f(M_{n+1})[s \in \mathbf{T}_2] \\
&= 
\int_A \mu_{\entrosp}(\dif s) \, \sum_{i \in \calI} f(E_i[\sigma_i(s)][\underline{\rho(s)}])[s \in U_i] \\
&= 
\sum_{i \in \calI} \int_{A \cap U_i} \mu_{\entrosp}(\dif s) \, f(E_i[\sigma_i(s)][\underline{\rho(s)}])\\
&\leq 
\sum_{i \in \calI} \int_{A \cap U_i} \mu_{\entrosp}(\dif s) \, f(E_i[\sigma_i(s)][\tsample])
\quad \hbox{$\because$ (\ref{eqn:a u ui})}\\
&= 
\int_{A} \mu_{\entrosp}(\dif s) \, \sum_{i \in \calI} \, f(E_i[\sigma_i(s)][\tsample])[s \in U_i]\\
&= \int_{A} \mu_{\entrosp}(\dif s) f_{n}(M)[s \in \mathbf{T}_2]
%\quad \hbox{$\because$ (\ref{eqn:f 2 n+1})}
\end{align*}

\end{proof}

As an immediate corollary of \Cref{lem:key rankable}, each $f(M_n)$ is integrable.
This concludes the proof of \Cref{thm:rankable and strict rankable}. 

\medskip

\newcommand\transform[1]{{\ulcorner{#1}\urcorner}}

%\section{$\tY$ stopping times are finite}
\iffalse
As before, fix a $M \in \Lambda^0$.
Recall the random variables on $(S, \calF, \mu)$:
\begin{align*}
T_0(s) & := 0 \\
T_{n+1}(s) & := \min \{ k \mid k>T_n(s), M_k(s) \textrm{ a value or of form } E[\tY \lambda x. N] \}
\end{align*}
\fi

\TnBounded*

We first show that $T_1$ is bounded i.e.~$T_1 \leq n$, for some $n \in \mathbb N$.

A first proof idea is to construct a reduction argument to the strong normalisation of the simply-typed lambda-calculus.
 There is a classical transform of conditionals into the pure lambda-calculus.
However, each evaluation of $\tsample$ (almost surely) returns a different number, which complicates such a transform.

Given a closed PPCF term $M$, we transform it to a term $\transform{M}$ of the nondeterministic simply-typed lambda-calculus as follows.
(We may assume that the nondeterministic calculus is generated from a base type $\iota$ with a $\iota$-type constant symbol $r$, and a function symbol $\bot : A$  for each function type $A$.)
\iffalse
\begin{enumerate}
\item replace every $\tsample$ by $r$ 

\item replace each base-type subterm $f \, M_1 \cdots M_n$ by $(\lambda x_1 \cdots x_n . r) \, \transform{M_1} \cdots \transform{M_n}$ where $f$ is a primitive function

\item replace every subterm $\tif{B}{M_1}{M_2}$ by
$(\lambda x . \transform{M_1} + \transform{M_2}) \, \transform{B}$, for a fresh $x$

\item replace every $\tY \, f \, x . N$ by $\lambda x . \transform{N}[\bot / f]$
\end{enumerate}
\lo{The above rules can be defined formally.}
\fi
\begin{align*}
\transform{\tsample} &:= r\\
\transform{y} &:= y \\
\transform{\lambda y . N} &:= \lambda y . \transform{N}\\
\transform{M_1 \, M_2} &:= \transform{M_1} \, \transform{M_2} \\
n \geq 0, \; \transform{f \, M_1 \cdots M_n} &:= (\lambda z_1 \cdots z_n . r) \, \transform{M_1} \cdots \transform{M_n} \\
\transform{\tif{B}{M_1}{M_2}} &:= \big(\lambda z . (\transform{M_1} + \transform{M_2})\big) \, \transform{B} \\
\transform{\tY N} &:= (\lambda y . \bot) \transform{N}
\end{align*}
In the above, variables $z, z_1, \cdots, z_n$ are assumed to be fresh; $f$ ranges over primitive functions and numerals.

The idea is that $\transform{M}$ captures the initial, non-recursive operational behaviour of $M$, so that $\transform{M}$ simulates the reduction of $M$ until the latter reaches a value, or a term of the form $E[\tY \lambda x. N]$.

It is straightforward to see that for every trace $s \in \entrosp$, there is a reduction sequence of $\transform{M}$ that simulates (an initial subsequene of) the reduction of $M$ under $s$.

Finally thanks to the following
\begin{theorem}[de Groote \cite{DBLP:conf/lfcs/Groote94}]
\label{thm:de groote}
The simply-typed nondeterministic lambda calculus is strongly normalising. \qed
\end{theorem}
we have:

\begin{enumerate}
\item Every $\transform{M}$-reduction terminates on reaching $r$, or a term of the shape $E[\bot N_1 \cdots N_n]$.

\item Further there is a finite bound (say $l$) on the length of such $\transform{M}$-reduction sequences; and $l$ bounds the stopping time $T_1$.
\end{enumerate}
This concludes the proof of \cref{lem:TnBounded}.

\iffalse
@inproceedings{DBLP:conf/lfcs/Groote94,
  author    = {Philippe de Groote},
  editor    = {Anil Nerode and
               Yuri V. Matiyasevich},
  title     = {Strong Normalization in a Non-Deterministic Typed Lambda-Calculus},
  booktitle = {Logical Foundations of Computer Science, Third International Symposium,
               LFCS'94, St. Petersburg, Russia, July 11-14, 1994, Proceedings},
  series    = {Lecture Notes in Computer Science},
  volume    = {813},
  pages     = {142--152},
  publisher = {Springer},
  year      = {1994},
  url       = {https://doi.org/10.1007/3-540-58140-5\_15},
  doi       = {10.1007/3-540-58140-5\_15},
  timestamp = {Tue, 14 May 2019 10:00:54 +0200},
  biburl    = {https://dblp.org/rec/conf/lfcs/Groote94.bib},
  bibsource = {dblp computer science bibliography, https://dblp.org}
}
\fi

\medskip

\TMtYStoppingTime*
\begin{proof}
%For each $n \in \omega$, $\set{T_M^\tY = n} \in \calF_{T_n}$.
To see $\set{T_M^\tY = n} \in \calF_{T_n}$ for a given $n$, we need to show that $\set{T_M^\tY = n} \cap \set{T_n \leq i} \in \calF_i$, for all $i \in \mathbb N$.
Let $\mathcal V$ be the set of values in $\Lambda^0$ (which is measurable), and let $\mathcal W$ be the non-values.
Then, it suffices to observe that $\set{T_M^\tY = n} \cap \set{T_n \leq i}
= \bigcup_{l=n}^i 
\big( M_l^{-1} [{\mathcal V}] \cap  
M_{l-1}^{-1} [{\mathcal W}] \cap 
\set{T_n = l} \big)
$, where each of $M_l^{-1} [{\mathcal V}]$, $M_{l-1}^{-1} [{\mathcal W}]$, and $\set{T_n = l}$ is in $\calF_i$.
\end{proof}



