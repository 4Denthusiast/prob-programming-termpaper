%%%%%%%%%%%%%%%%%%%%%%%%%%
%% BEGIN {Introduced by Luke}

\usepackage{paralist}

% \usepackage{appendix}
%% END {introduced by Luke}
%%%%%%%%%%%%%%%%%%%%%%%%%%

\usepackage[inline]{enumitem}
\setlist[enumerate,1]{label={(\roman*)}}
\usepackage{bussproofs}

\usepackage[caption=false]{subfig}

\newlist{thmlist}{enumerate}{1}
\setlist[thmlist]{label=\textup{(\roman{thmlisti})},ref={(\roman{thmlisti})},noitemsep}

\newlist{complist}{enumerate}{1}
% in that case, at least label must be specified using \setlist
\setlist[complist,1]{noitemsep,label=\textbf{(C\arabic*)},leftmargin=2.5\parindent}

\usepackage{diagbox} %for diagonal division of a cell in tabular
\usepackage{makecell} %for diagonal division of a cell in tabular

\usepackage{pifont}% http://ctan.org/pkg/pifont
\newcommand{\cmark}{\ding{51}}% tick
\newcommand{\xmark}{\ding{55}}% cross

\tikzstyle{resnode}=[anchor=base,fill=black!15,inner sep=2pt]
\tikzstyle{unode}=[anchor=base]

\DeclareMathOperator{\qc}{qc}
\DeclareMathOperator{\qm}{qm}
%\ifdraft

\makeatletter

\renewcommand{\p@thmlisti}{\perh@ps{\thetheorem}}
\protected\def\perh@ps#1#2{\textup{#1#2}}
\newcommand{\itemrefperh@ps}[2]{\textup{#2}}
\newcommand{\itemref}[1]{\begingroup\let\perh@ps\itemrefperh@ps Part~\ref{#1}\endgroup}
\newcommand{\itemrefs}[2]{\begingroup\let\perh@ps\itemrefperh@ps Parts~\ref{#1} and~\ref{#2}\endgroup}

% \renewcommand{\Res}{\Rightarrow_{\Resh,\As}}
% \newcommand{\Resp}{\Rightarrow_{\Reshh,\Ad}}
% \newcommand{\Respstr}[1]{\Rightarrow_{\Reshh,#1}}
\renewcommand{\Res}{\vdash_\As}
\newcommand{\Resp}{\vdash_\Ad}
\newcommand{\Respstr}[1]{\vdash_{#1}}
\newcommand{\embed}[1]{\left[ #1\right]}


\newcommand{\vals}{\mathfrak a}
% \renewcommand{\Rd}{\mathfrak r}
% \renewcommand{\Sd}{\mathfrak s}
% \renewcommand{\Td}{\mathfrak t}

% theorems
\declaretheorem[
    name=Theorem,
    Refname={Theorem,Theorems},
    numberwithin=section]{theorem}
\declaretheorem[
    name=Lemma,
    Refname={Lemma,Lemmas},
    sibling=theorem]{lemma}
\declaretheorem[
    name=Proposition,
    Refname={Prop.,Propositions},
    sibling=theorem]{proposition}
\declaretheorem[
    name=Corollary,
    Refname={Corollary,Corollaries},
    sibling=theorem]{corollary}
\declaretheorem[
    name=Claim,
    Refname={Claim,Claims}
    ]{claim}
\declaretheorem[
    name=Fact,
    Refname={Fact,Facts},
    sibling=theorem]{fact}
\declaretheorem[
    name=Conjecture,
    Refname={Conjecture,Conjectures},
    sibling=theorem]{conjecture}
\declaretheorem[numbered=no,
    name=Assumption,
    Refname={Assumption,Assumptions}]{assumption}
\theoremstyle{definition}
\declaretheorem[
    name=Definition,
    Refname={Definition,Definitions},
    sibling=theorem]{definition}
\declaretheorem[
    name=Example,
    Refname={Example,Examples},
    sibling=theorem]{example}
\theoremstyle{remark}
\declaretheorem[
    name=Remark,
    Refname={Remark,Remarks},
    sibling=theorem]{remark}

    \declaretheoremstyle[
    spaceabove=-4pt, 
    spacebelow=6pt, 
    headfont=\it, 
    bodyfont = \normalfont,
    postheadspace=1em, 
    qed=$\blacksquare$, 
    headpunct={.}]{myproofstyle} %<---- change this name
\declaretheorem[name={Proof}, style=myproofstyle, unnumbered]{claimproof}

% reset claim counter at the beginning of each proof
\let\oldproof\proof
\let\oldendproof\endproof
\def\proof{\setcounter{claim}{0}\begingroup\oldproof}
\def\endproof{\oldendproof \endgroup}


% \declaretheoremstyle[
%     spaceabove=-4pt, 
%     spacebelow=6pt, 
%     headfont=\it, 
%     bodyfont = \normalfont,
%     postheadspace=1em, 
%     qed=$\blacksquare$, 
%     headpunct={.}]{myproofstyle} %<---- change this name
% \declaretheorem[name={Proof}, style=myproofstyle, unnumbered]{claimproof}

% % reset claim counter at the beginning of each proof
% \let\oldproof\proof
% \let\oldendproof\endproof
% \def\proof{\setcounter{claim}{0}\begingroup\oldproof}
% \def\endproof{\oldendproof \endgroup}

\usepackage{hyperref}
% \fi
\usepackage[capitalize]{cleveref}

%\usepackage{cleveref}

\Crefname{theorem}{Thm.}{Theorems}
\Crefname{corollary}{Cor.}{Corollary}
\crefname{proposition}{Prop.}{Propositions}
\Crefname{claim}{Claim}{Claims}
\Crefname{definition}{Def.}{Definitions}
\Crefname{fact}{Fact}{Facts}
\Crefname{conjecture}{Conj.}{Conjectures}
\Crefname{example}{Ex.}{Examples}
\Crefname{remark}{Rem.}{Remarks}
\Crefname{convention}{Convention}{Conventions}
\Crefname{lemma}{Lem.}{Lemmas}
%\Crefname{assumption}{Assumption}{Assumptions}
\Crefname{section}{Sec.}{Sec.}
\Crefname{appendix}{App.}{App.}
\Crefname{figure}{Fig.}{Fig.}

\addtotheorempostheadhook[theorem]{\crefalias{thmlisti}{theorem}}
\addtotheorempostheadhook[lemma]{\crefalias{thmlisti}{lemma}}
\addtotheorempostheadhook[proposition]{\crefalias{thmlisti}{proposition}}
\addtotheorempostheadhook[corollary]{\crefalias{thmlisti}{corollary}}
\addtotheorempostheadhook[claim]{\crefalias{thmlisti}{claim}}
\addtotheorempostheadhook[fact]{\crefalias{thmlisti}{fact}}
\addtotheorempostheadhook[conjecture]{\crefalias{thmlisti}{conjecture}}
\addtotheorempostheadhook[example]{\crefalias{thmlisti}{example}}
\addtotheorempostheadhook[definition]{\crefalias{thmlisti}{definition}}
\addtotheorempostheadhook[remark]{\crefalias{thmlisti}{remark}}

\newcommand{\inferbinlics}[5]{\begin{minipage}{22ex}{\bfseries #1}\end{minipage} {
    \begin{minipage}{10ex}{$\infer{#4}{#2 &&& #3}$}\end{minipage}
    \par\noindent #5}}
\newcommand{\inferthrlics}[6]{\begin{minipage}{13ex}{\bfseries #1}\end{minipage} {
    \begin{minipage}{10ex}{$\infer{#5}{#2 &&& #3 &&& #4}$}\end{minipage}
    \par\noindent #6}}
\newcommand{\inferunlics}[4]{\begin{minipage}{22ex}{\bfseries #1}\end{minipage} {
    \begin{minipage}{10ex}{$\infer{#3}{#2}$}
    \end{minipage}
    \par\noindent
      #4
    }}
\newcommand{\inferannlics}[5]{\begin{minipage}{22ex}{\bfseries #1}\end{minipage} {
    \begin{minipage}{10ex}{$\infer[#3]{#4}{#2}$}\end{minipage}
    \par\noindent #5}}

\newcommand{\inferthrplics}[6]{\begin{minipage}{22ex}{\bfseries #1}\end{minipage} {
    \begin{minipage}{10ex}{$\infer{#5}{#2 &&& #3 &&& #4}$}\end{minipage}
    \par\noindent #6}}




% Set high/infinite penalties for inappropriate line breaking.
% \binoppenalty=\maxdimen
% \relpenalty=\maxdimen
% \clubpenalty10000
% \widowpenalty10000
% \displaywidowpenalty=10000


\makeindex
