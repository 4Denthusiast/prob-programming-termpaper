% !TEX root = main.tex
\section{Related work and further directions}
\label{sec:related}

\lo{TODO. Related work (.75 column). Discuss \cite{DBLP:conf/birthday/BlancLM05}.} \akr{I think that that labelling scheme is not really that related. Would a note like "This is another scheme for labelling positions in reachable terms, therefore it appears superficially similar, but it's not actually that related." actually be worthwhile? I suppose it could at least reassure readers that we're aware of that paper and it isn't the same thing, but I don't know how likely readers are to actually draw the relation in the first place.}

\lo{Our results specialise to proving termination of non-random lambda calculus, which is much studied. Our approach is closest in spirit to Jones' work on size-change principle \cite{DBLP:journals/lmcs/JonesB08,DBLP:conf/aplas/SereniJ05}. Dal Lago's \cite{DBLP:journals/toplas/LagoG19} work builds on this. We think that martingales are a good framework for generalising / unifying with size-change work.}

\lo{Other approaches, mainly based on types, are complementary to ours.}

\lo{Kobayashi et al.: PHORS are strictly weaker in expressivity \cite{DBLP:conf/lics/KobayashiLG19}.}

The main theorem in \cite{DBLP:journals/pacmpl/McIverMKK18} is similar to our \ref{thm:antitone rankable implies termination}, but in an imperative language. While we require that the antitone ranking function decreases for every $\tY$-reduction step, they require that it decreases for every iteration of a certain while loop (the ranking function in that case is defined in the context of a particular loop), which is similar to the recursion construct $\tY$ but limited to tail recursion. The difference in the exact progress condition is not significant. Their progress condition, that there are antitone functions $p$ and $d$ such that the ranking function decreases from $x$ by at least $d(x)$ with probability at least $p(x)$, would not have worked directly because it does not cope as easily with the fact that the ranking function may change in value at the intermediate non-$\tY$ reduction steps (provided it does not increase in expectation). However, it is easy enough to convert a ranking function that satisfies either progress condition to the other.

The confluent semantics, and the results based on it, is new in the setting of a functional language, because an imperative language does not have any equivalent of other redexes, or a similar nondeterministic structure. The sparse ranking function construction is also more useful in a functional language (although it would be possible to give an equivalent in an imperative language), because in an imperative language where the order of execution is more rigid, the iterations of the while loop provide a natural set of checkpoints where it is reasonable to give values of the ranking function.

We have not yet been able to prove \Cref{conj:antitone} which would imply that \Cref{thm:antitone rankable implies termination} is almost complete.
\begin{conjecture}
\label{conj:antitone}
Every term is antitone rankable if all of the terms reachable from it are AST.
\end{conjecture}

The \Cref{def:more general red} of redexes and which positions are acceptable to reduce at is sufficiently restrictive to guarantee Church-Rosserness, but is also a little more restrictive in some cases than is necessary for this purpose. For example, if the argument of a function is not yet a value, but its reduction to a value would be deterministic, or the function is affine, then applying the function before reducing its argument would not cause any problematic duplication of random samples. Similarly, if $\tsample$ occurs at position $\alpha;@_2;\beta;\lambda;\gamma$, but the function at $\alpha;@_1$ is affine, evaluating the $\tsample$ may in some cases also work fine. A more complete characterisation of which redexes can be reduced without breaking Church-Rosserness could be interesting.

Although this is a broadly applicable method of proving almost sure termination of probabilistic functional programs, and \Cref{thm:antitone partial implies rankable,thm:confluent ranking} make it more convenient to use, some method of automating the construction and checking of (antitone) ranking functions, even partially, could make it considerably more practically useful, especially in cases where almost sure termination is merely a side-condition for some other theorem or algorithm to be applicable.
