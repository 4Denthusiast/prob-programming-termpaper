% !TEX root = main.tex

\begin{align*}
  \text{\em types } A, B &::= \textsf{R}  \mid  A \to B \\
  \text{\em values } V &::= x \mid \lambda x.M  \mid  \underline{r} \mid \tY \\
  \text{\em terms } M, N &::= V  \mid  M_1 \, M_2  \mid  \underline{f}(M_1,\dots ,M_n) \\
   & \qquad \mid  \tif{M < 0}{N_1}{N_2}  \mid  \tsample
\end{align*}
Currently $\tY$ is not a legal term. This is unnatural: there is no reason to so restrict $\tY$. Following \citep[p.~376]{DBLP:conf/fsttcs/Sieber90}, we should define $\tY$ as a value (and so a proper term); it follows that $\tY \, M$ is redundant as a clause.

It is also standard to regard variables as values in a CBV language.

Problem: $\tY \, \tY$ is then a ``stuck'' term: neither value nor redex. A possible solution is to redefine the $\tY$-redex rule as $\tY \, V \to \lambda z . V \, (\tY \, V) \, z$, but I expect this would affect many proofs in the confluent semantics section, and there may not be time for such a change.

\begin{example}[Non-affine recursion with continuous distribution]
\label{ex:non-affine with continuous distribution}
Consider 
\[
\Xi := \tY \lambda f x . \big(\mathsf{let} \; e = \tsample \; \mathsf{in} \;
\mathsf{if}\boldsymbol{(}e \leq p, x, X\boldsymbol{)}\big)
\]
where
\(
X = \big( f^3(x+1) \oplus_e f^2(x+1) \big) \oplus_{\mathit{sig}(x)} f^2(x+1) 
\)
and $\mathit{sig}(x)$ is the sigmoid function.

Not only is this program non-affine recursive,
note also the use of a random sample, $e$, as a first-class value, and the subsequent use as a probability in the binary choice $\oplus_e$.
Such a computation cannot be modelled via discrete distributions. 

We can use the ranking function method (coupled with the solution of linear recurrence relations) to show that provided $p \geq \sqrt{7} - 2$, the program $\Xi \, \underline r$ is AST for all $r \in \Real$.

\lo{This example is taken from the series of 3d-printing-company running examples (Beutner-Ong 2021).
This paper is as yet unpublished, and effectively uncitable for the time being (because of strict double-blind submission rules).}
\end{example}
