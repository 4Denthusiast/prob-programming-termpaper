% !TEX root = main.tex
\section{Syntax of Probabilistic PCF}
\label{sec:PPCF}

The language PPCF is a call-by-value version of PCF with sampling of real numbers from the closed interval $[0,1]$ \cite{Ehrhard2018c,DBLP:journals/pacmpl/EhrhardPT18,MakOPW21}.
%\changed[lo]{following \cite{MakOPW21} except the score function (\Cref{rem:score})}. 
Types and terms are defined as follows, where $r$ is a real number, $x$ is a variable, $f : \mathbb{R}^n \to \mathbb{R}$ is any measurable function, and $\Gamma$ is an environment:
\begin{align*}
  \text{\em types } A, B &::= \textsf{R}  \mid  A \to B \\
  \text{\em values } V &::= \lambda x.M  \mid  \underline{r} \\
  \text{\em terms } M, N &::= V  \mid  x  \mid  M_1 \, M_2  \mid  \underline{f}(M_1,\dots ,M_n)  \mid  \tY \, M \\
   & \qquad \mid  \tif{M < 0}{N_1}{N_2}  \mid  \tsample
\end{align*}
The typing rules are standard (see \Cref{fig:typing rules}). The restriction to well-typed terms is only necessary here in order to avoid reaching terms which contain nonsense such as applying a number as though it were a function, so a more liberal type system would work just as well. Simple types are just used for simplicity.
Terms are identified up to $\alpha$-equivalence, as usual. 
The set of all terms is denoted $\Lambda$, and the set of closed terms is denoted $\Lambda^0$.

\begin{figure*}[htb]
\begin{align*}
  \frac{}{\Gamma ; x:A \vdash x:A} 
  \qquad
  \frac{\Gamma ; x:A \vdash M : B}{\Gamma \vdash \lambda x.M : A \to B} 
  \qquad
  \frac{\Gamma \vdash M:A \to B \quad \Gamma \vdash N : A}{\Gamma \vdash M \, N : B} 
  \\ \\
  \qquad
  \frac{\Gamma \vdash M : (A \to B) \to (A \to B)}{\Gamma \vdash \tY M : (A \to B)}
  \qquad
  \frac{\Gamma \vdash M : \textsf{R} \quad \Gamma \vdash N_1 : A \quad \Gamma \vdash N_2 : A}{\Gamma \vdash \tif{M < 0}{N_1}{N_2} : A}    
  \\ \\
  \qquad
  \frac{}{\underline{r} : \textsf{R}} 
  \qquad
  \frac{}{\Gamma \vdash \tsample : \textsf{R}}
  \qquad
  \frac{\Gamma \vdash M_1:\textsf{R} \quad \dots \quad \Gamma \vdash M_n:\textsf{R}}{\Gamma \vdash \underline{f}(M_1,\dots,M_n) : \textsf{R}} \ (f : \mathbb{R}^n \to \mathbb{R})
\end{align*}
\caption{Typing rules of PPCF \label{fig:typing rules}}
\end{figure*}

