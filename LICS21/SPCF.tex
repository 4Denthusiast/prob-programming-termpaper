% !TEX root = main.tex
\section{Statistical PCF}
\label{sec:SPCF}

\subsection{Syntax of SPCF}

The language SPCF is a simply-typed lambda calculus with sampling of real numbers from the closed interval $[0,1]$ and unbounded scoring, following \cite{MakOPW21}. 
Types and terms are defined as follows, where $r$ is a real number, $x$ is a variable, $f : \mathbb{R}^n \to \mathbb{R}$ is any measurable function, and $\Gamma$ is an environment:
\begin{align*}
  \text{\em types } A, B &::= \textsf{R}  \mid  A \to B \\
  \text{\em values } V &::= \lambda x.M  \mid  \underline{r} \\
  \text{\em terms } M, N &::= V  \mid  x  \mid  M_1 \, M_2  \mid  \underline{f}(M_1,\dots ,M_n)  \mid  \tY M \\
   & \qquad \mid  \tif{M < 0}{N_1}{N_2}  \mid  \tsample  \mid  \tscore(M)
\end{align*}
The typing rules are standard (see \Cref{fig:typing rules}).
Terms are identified up to $\alpha$-equivalence, as usual. 
The set of all terms is denoted $\Lambda$, and the set of closed terms is denoted $\Lambda^0$.

%\paragraph{}
To define the reduction relation, let \emph{evaluation contexts} be of the form:
\begin{align*}
  E ::= & \, [\,] \mid E \, M \mid V \, E \mid \underline{f}(\underline r_1,\dots ,\underline r_{k-1}, E, M_{k+1}, \dots, M_n) \\ & \mid \tY E \mid \tif{E<0}{M_1}{M_2} \mid \tscore (E)
\end{align*}
then a term reduces if it is formed by substituting a redex in a context i.e.
\begin{align*}
  E[(\lambda x.M) V] & \to E[M[V/x]] \\
  E[\underline f (\underline r_1, \dots , \underline r_n)] & \to E[\underline{f(r_1,\dots,r_n)}] \\
  E[\tY \lambda x. M] & \to E[\lambda z. M[(\tY \lambda x. M)/x] z] \\
  & \qquad \qquad \text{ where $z$ is not free in $s$}\\
  E[\tif{\underline r < 0}{M_1}{M_2}] & \to E[M_1] \text{ where }r < 0 \\
  E[\tif{\underline r < 0}{M_1}{M_2}] & \to E[M_2] \text{ where }r \geq 0 \\
  E[\tsample] & \to E[\underline r] \text{ where } r \in [0,1] \\
  E[\tscore(\underline r)] & \to E[\underline r].
\end{align*}
We write $\to^\ast$ for the reflexive, transitive closure of $\to$.
\lo{The $\tscore$ rule should have a side condition: if $r > 0$.}

Every closed well-typed term either is a value or reduces to another term.

\subsection{Sampling semantics}
\label{sec:sampling semantics}
This version of the reduction relation allows $\tsample$ to reduce to any number in $[0,1]$. 
To more precisely specify the probabilities, an additional argument is needed to determine the outcome of random samples. Let $ I := [0,1] \subset \mathbb{R} $, and define $\entrosp := I^{\mathbb{N}}$, with the \changed[lo]{Borel $\sigma$-algebra $\Sigma_\entrosp$} and the probability measure, denoted $\mu_{\entrosp}$, given by the limit of $1 \gets I \gets I^2 \gets \cdots$, where the maps are the projections that ignore the last element. Equivalently, a basis of measurable sets is $\prod_{i=0}^\infty X_i$ where $X_i$ are all Borel and all but finitely many are $I$, and $\mu_{\entrosp} (\prod_{i=0}^\infty X_i) := \prod_{i=0}^\infty \Leb(X_i)$.
The maps $\pi_h:\entrosp \to I, \; \pi_t:\entrosp \to \entrosp$ popping the first element are then measurable.
\changed[lo]{Following \cite{DBLP:conf/esop/CulpepperC17}, we call the probability space $(\entrosp, \Sigma_\entrosp, \mu_{\entrosp})$ the \emph{entropy space}.}

The $\sigma$-algebra and measure on $\Lambda$ are defined by considering it as a disjoint union of equivalence classes under replacing all the real constants by a placeholder, where the measure on each class is that of $\mathbb{R}^n$, where $n$ is the number of real constants.
\changed[lo]{Precisely, following \cite{DBLP:conf/icfp/BorgstromLGS16},
we view $\Lambda$ as $\bigcup_{m\geq 0} \big(\mathsf{Sk}_m \times \Real^m \big)$,
where $\mathsf{Sk}_m$ is the set of SPCF terms with exactly $m$ place-holders (a.k.a.~\emph{skeleton terms}) for numerals.
Thus identified, we give $\Lambda$ the countable disjoint union topology of the product topology of the discrete topology on $\mathsf{Sk}_m$ and the standard topology on $\Real^m$.
Note that the connected components of $\Lambda$ have the form $\{M\} \times \Real^m$, with $M$ ranging over $\mathsf{Sk}_m$, and $m$ over $\mathbb{N}$. 
We fix the Borel algebra of this topology to be the $\sigma$-algebra on $\Lambda$.}

The one-step reduction is given by the function $\red : \Lambda^0 \times \entrosp \to \Lambda^0 \times \entrosp$ where
\begin{equation*}
\red(M,s) := \left\{
    \begin{array}{ll}
        (E[N],s) & \text{if } M = E[R], R \to N\\
        & \qquad \text{ and } R \neq \tsample \\
        (E[\underline{\pi_h(s)}],\pi_t(s)) & \text{if } M = E[\tsample] \\
        (M,s) & \text{if } M \text{ is a value}
    \end{array} \right .
\end{equation*}
\lo{Both $\to$ and $\red$ are called reduction relation.}

The result after $n$ steps is then simply 
\[
\red^n(M,s) = \underbrace{\red(\ldots\red(}_n M,s)\ldots)
\] 
and the limit $\red^\infty$ can then be defined as a partial function as $\lim_{n \to \infty} \red^n(M,s)$ whenever that sequence becomes constant by reaching a value. A term $M$ terminates for a sample sequence $s$ if the limit $\red^\infty(M,s)$ is defined.

The reduction function is measurable, and the set of values is measurable, therefore the set of $s$ such that $M$ terminates at $s$ within $n$ steps is measurable for any $n$, therefore $\{s \mid M \text{ terminates at } s \}$ is measurable. 
We say that a term $M$ is \emph{almost-surely terminating} (AST) just if $\mu_{\entrosp}(\{s \mid M \text{ terminates at } s\}) = 1$.

\iffalse
\lo{Alternatively, define the \emph{runtime of $M$} to be the random variable 
\[
T_M(s) := 
\begin{cases}
\min \set{n \mid \pi_0(\red^n(M, s)) \textrm{ is a value}} & \hbox{if $\red^\infty(M,s)$ is defined}\\
\infty & \hbox{otherwise}
\end{cases}
\]
Equivalently, we say that $M$ is \emph{almost-surely terminating} (AST) if $T_M < \infty$ a.s.; 
and $M$ is \emph{positively almost-surely terminating} (PAST) if $\expect{T_M} < \infty$.}
\fi
For example, the term 
\[
(\tY \lambda f, n: \tif{\tsample - 0.5 < 0}{n}{f (n+1)} ) \, \underline{0},
\] 
which generates a geometric distribution, terminates on the set $\entrosp \setminus [0.5,1]^\mathbb N$, which has measure 1, therefore it terminates almost surely, whereas 
\[
\tif{\tsample - 0.5 < 0}{\underline 0}{(\tY \lambda x. x) \underline 0}
\] 
which terminates on the set $\pi_h^{-1}[[0,0.5]]$, has probability 0.5 of failing to terminate.

%\paragraph{}
Note that the requirement that $M$ be closed is not important for the proofs of the theorems, but is simply required to ensure that $M$ reaches a value, rather than a general normal form like $x \, y$.

%\paragraph{}
This definition of almost-sure termination is equivalent to that given in \citep{MakOPW21}, although the program semantics is stated in a slightly different way. In particular, the argument to $\tscore$ is not relevant to termination (except for the possibility that its argument's evaluation wouldn't terminate).

\iffalse
\lo{Your operational semantics does not maintain a record of the current weight of the reduction.
A.s.~termination does depend on $\tscore$: see \cite[\S 4.3]{MakOPW21}\footnote{\url{https://arxiv.org/abs/2004.03924}}.
I think it important to take the behaviour of $\tscore$ into account;
you should do it as a future task.}
\fi
