\documentclass[titlepage]{article}
\usepackage{amsmath}
\usepackage{amsfonts}
\newcommand{\tY}{\mathsf{Y}}
\newcommand{\tsample}{\mathsf{sample}}

\title{DPhil Thesis Proposal: Semantics of Probabilistic Functional Languages}
\author{Andrew Kenyon-Roberts}

\begin{document}
\maketitle
\section{Introduction}


%Existing approaches:
    %brief summary, axes of difference: operational, denotational, distribution-based, trace-based
    %Operational Semantics
    %Denotational Semantics
        %Denotational Semantics in the non-probabilistic case, types and inconsistencies
        %QBSes
    %Distribution-based Semantics
    %Trace-based Semantics

\section{My Previous Research}
My thesis will be building on the paper I recently submitted to LICS (in collaboration with Luke Ong), Supermartingales, Ranking Functions and Probabilistic Lambda Calculus. In it, we extend existing work on proving almost-sure termination using ranking functions (functions from the program state to non-negative real numbers which at each step have a non-positive expected change, and which satisfy some further progress condition) to the context of a functional language with continuous distributions. Because the language used is simply-typed, the only way that non-termination is possible is using the explicit recursion contruct, $\tY$, therefore adding a condition that the ranking function must decrease by 1 for every $\tY$-reduction step provides a bound on the expected number of $\tY$-reduction steps, which implies almost sure termination. This result is then extended in three ways (which can all be combined).

The first extension is sparse ranking functions. A sparse ranking function may be defined at a subset of the reachable states of the program, with the condition instead being that the expected value of the ranking function at the next step where it is defined is not greater than the current value (and, if there are intermediate $\tY$-reduction steps, there is a corresponding strict decrease). Although this does not increase the strength of the technique overall, it makes it easier to apply, because the ranking function only needs to be defined at those points in the program's execution which are actually interesting. This extension is new, having not been applicable in the context of first-order languages because the completed body of a while loop provides a natural set of checkpoints.

The second extension is antitone ranking functions. This idea is taken from \cite{some citation needed}, just adapted to the new functional context. A function $\epsilon : \mathbb R_{\geq 0} \to \mathbb R_{> 0}$, which is required to be monotoically decreasing, is added to the definition, and rather than the ranking function having to decrease by 1 at every $\tY$-reduction step, it only has to decrease by $\epsilon$ of the current value of the ranking function. This allows the termination of programs which terminate much more slowly (such as the unbiased random walk) to be proven, whereas the basic result was only applicable to terms which terminated with a finite expected number of $\tY$-reduction steps.

The third extension, and the most interesting and relevant to my intended thesis topic, is ranking functions with respect to alternative reduction strategies. This result is proven using a variant of trace semantics which is able to satisfy a restricted confluence property. Probabilistic functional languages in general do not have the same sort of confluence properties as pure lambda calculus. For example, assuming that samples are drawn from the uniform distribution on $[0,1]$, the term $(\lambda x. x + x) \tsample$ evaluates either to the uniform distribution on $[0,2]$ or to the triangle-shaped distribution with the same support, with its peak at 1, depending on whether the sample is evaluated first (in which case, one value is doubled) or the $\beta$-redex is evaluated first (in which case, two independent samples are taken and added). If, however, certain reductions are excluded (in the results I proved in this paper, it's $\beta$-reductions whose arguments aren't values, and $\tsample$-reductions inside $\lambda$s either inside $\tY$s or in the argument of applications, although other criteria would also be possible), confluence can be re-gained.

This is enough to ensure that the distribution of results is (barring cases where some reduction strategies may terminate while others don't) the same, but that would be difficult to prove directly, and the result doesn't extend to the same traces giving the same results. For example, if the trace is simply a sequence of samples which are used in the order in which the $\tsample$-reductions occur, then the term $\tsample - \tsample$ reduces with the trace $0, 1, ...$ to either $1$ or $-1$, depending on which $\tsample$ is reduced first. This example can be fixed by labelling the samples in the trace by positions in the term rather than the order in which they're used, but in other cases, that would still not be sufficient. Consider, for example, $(\lambda f. f 0 + f 0) (\lambda x. \tsample)$. This reduces (by some reduction strategies) to $\tsample + \tsample$, but the $\tsample$s here don't correspond in a unique way to the $\tsample$s in the initial term, therefore labelling samples by positions in the initial term isn't sufficient. Perhaps it would be possible to in some way label those $\tsample$s by a combination of the position of the $\tsample$ they're derived from and the occurrences of the variable $f$ that they were substituted in to, but the approach I ended up taking was a somewhat more brute-force solution, in that if any labelling scheme would have worked, so would this one, by design. Samples are labelled by a combination of a term reachable from the initial term (technically a reachable skeleton instead, with all the real numbers removed, for measure-theoretic reasons) and a position within that term, with an equivalence relation defined on these (reachable term, position) pairs that is the minimum required to ensure confluence.

This definition is sufficient to prove the AST result that is nominally the main focus of that paper, but also seems like something that has a lot of potential to be used for other purposes. The limits of confluence for probabilistic functional languages is not something I have seen discussed extensively elsewhere, except the fact that some reduction strategies (specifically call-by-name and call-by-value) give different results. The way that confluence is obtained with a trace-based semantics also seems likely to extend to a denotational trace semantics.

\section{Planned research}
\subsection{Using Labelling to Combine Denotational and Trace Semantics}
Trace-based semantics is, in a sense, finer than distribution-based semantics. Given a measurable function $f : \Lambda \times \mathbb S \to A$ that describes some property of a term $\in \Lambda$ by using a trace from some sampling space $\mathbb S$, which has some fixed probability measure on it $\mu_{\mathbb S}$, this can be converted to a kernel from $\Lambda$ to $A$ using the push-forward measure of $\mu_{\mathbb S}$, but some information is lost in the process: which specific trace lead to which outcome. There is no corresponding simple way to convert a distribution-based semantics to a trace-based semantics that actually results in all the correct structure expected of a trace-based semantics. In trace-based semantics, it is possible to define things like correlations between the random variables it defines, because they are all defined in the same probability space, whereas this is not possible given merely a distribution of outputs.

There are also some advantages of denotational over operational semantics. It provides meanings to sub-programs that can be reasoned about independently of their context. It can be more suitable for proving things like the effectiveness of a type system for proving termination, as the structure of the semantics more closely matches the derivations of typing judgements.%This paragraph could really do with some references. They don't even have to be probabilistic, just demonstrations of how useful denotational semantics in general can be.

Existing approaches are either denotational and distribution-based, operational and distribution-based, or operational and trace-based. The combination of denotational and trace-based is more difficult to define, because some way of distinguishing multiple copies of the same sample statement is needed, so that they can derive their randomness from different parts of the trace. In operational semantics, this is easy enough, because the argument of a beta redex does not need to be evaluated at all before it is duplicated by the reduction, and its copies can simply be dealt with one at a time, like any other independent subterms. In denotational semantics, it is not the terms themselves which are combined in an application, but their interpretations, therefore the interpretation of the argument needs some way to retain the possibility of having multiple different results from the random process, even though there is only one trace used overall.

An extension of the labelling scheme used in the confluent trace semantics of my previous paper could provide just such a possibility. The exact labelling scheme used there depends explicitly on the reduction sequences (and is also rather convoluted), so it is not suitable to use directly in defining a denotational semantics, but it seems highly likely that some labelling scheme that's equivalent but differently defined would be. If all of the subterms of the original term are given unique labels, and the labels form some sort of magma, so that when a reduction occurs, the labels of the argument's subterms and the function's subterms can be combined in some way to produce the labels of the reduct's subterms, then a trace that contains a value for every possible combination of labels could be used. The random samples taken in the evaluation of each subterm then depend on the transformation of the labels that occurs, which is distinct for each copy of the subterm.

\subsection{Using labelling to perform efficient inference}
Another possible application of a labelling scheme for sample is in improving the efficiency of certain inference algorithms. In lightweight metropolis-hastings, the program is first executed, picking random values for all the samples taken, with the trace of samples being remembered. Then, random variations on the trace are tested, re-running the program with the modified trace and accepting or rejecting the modification based on the difference in the likelihood of the runs. It is essential to the efficiency of this algorithm that the score of a modified run is highly correlated to the score of the original run, as otherwise the acceptance ratio becomes low. At the same time, it is necessary that the proposed changes actually change the trace by not too small an amount, so that the distribution of traces converges to the stationary distribution quickly.

The correlation between runs can be impeded if the number of samples taken varies depending on the results of earlier random choices. In the basic version of the algorithm, the trace is simply a list of the samples in the order in which they are taken. If some sample is never used, however, the rest of the samples in the list become misaligned, and end up taking different roles in the resulting execution, completely changing the score with only a small change to the trace.

This, then, is another possible application of a labelling scheme similar to the one previously mentioned. By better approximating the roles that samples play in the program execution by the labels, the problem of trace misalignment can be reduced. In \cite{lee2019towards}, a similar approach is taken for variational inference with a first-order language, but in that case, they require that the labels be explicitly provided in the program (potentially being computed using the values of other variables). There is also a similar proposal in \cite{pmlr-v15-wingate11a}.

The requirements of a labelling scheme for this purpose are somewhat different from the previous cases. In the other cases, the labels needed to match across program executions that differ nondeterministically in their reduction order, but for the purposes of inference, the relevant different executions differ by the randomly selected samples. A labelling scheme adapted to the problem of inference could probably be simplified relative to the confluent trace semantics defined in my previous paper, or the more algebraic variant I intend to develop for the denotational trace semantics mentioned earlier. I am not yet sure how similar it would end up, after this simplification, to the much simpler labelling scheme described in \cite{pmlr-v15-wingate11a}.

\subsection{Completeness of Antitone Rankability}
A term is rankable iff every term reachable from it terminates with finitely many $\tY$-reduction steps in expectation. This essentially puts a limit on how slowly a term can terminate while still being rankable, so any term which is not $\tY$-positive almost surely terminating cannot be proven AST by assigning it a ranking function. With antitone ranking functions, there is no analogous restriction: there are arbitrarily slowly terminating terms (i.e. for any probability distribution on $\mathbb N$, there is a term whose distribution of number of $\tY$-reduction steps before termination is the given distribution) which are antitone rankable. It therefore seems rather likely that any term such that every term reachable from is is AST is antitone rankable. I intend to try to prove this result, even though it does not relate very closely to my overall thesis topic, because it would make my existing work more complete.
\end{document}
