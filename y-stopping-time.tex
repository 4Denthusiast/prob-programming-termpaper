% !TEX root=./probabilisticProgrammingMartingales.tex

\newcommand\transform[1]{{\ulcorner{#1}\urcorner}}

\section{$\tY$ stopping times are finite}

As before, fix a $M \in \Lambda^0$.
Recall the random variables on $(S, \calF, \mu)$:
\begin{align*}
T_0(s) & := 0 \\
T_{n+1}(s) & := \min \{ k \mid k>T_n(s), M_k(s) \textrm{ a value or of form } E[\tY \lambda x. N] \}
\end{align*}

\begin{lemma}
$T_0, T_1, T_2, \cdots$ are an increasing sequence of stopping times adapted to $(\calF_n)_{n \geq 0}$, and each $T_i$ is bounded.
\end{lemma}

We first show that $T_1$ is bounded i.e.~$T_1 \leq n$, for some $n \in \omega$.

A first proof idea is to construct a reduction argument to the strong normalisation of the simply-typed lambda-calculus.
 There is a classical transform of conditionals into the pure lambda-calculus.
However, each evaluation of $\tsample$ (almost surely) returns a different number, which complicates such a transform.

Given a closed SPCF term $M$, we transform it to a term $\transform{M}$ of the nondeterministic simply-typed lambda-calculus as follows.
(We may assume that the nondeterministic calculus is generated from a base type $\iota$ with a $\iota$-type constant symbol $r$, and a function symbol $\bot : A$  for each function type $A$.)
\iffalse
\begin{enumerate}
\item replace every $\tsample$ by $r$ 

\item replace each base-type subterm $f \, M_1 \cdots M_n$ by $(\lambda x_1 \cdots x_n . r) \, \transform{M_1} \cdots \transform{M_n}$ where $f$ is a primitive function

\item replace every subterm $\tif{B}{M_1}{M_2}$ by
$(\lambda x . \transform{M_1} + \transform{M_2}) \, \transform{B}$, for a fresh $x$

\item replace every $\tY \, f \, x . N$ by $\lambda x . \transform{N}[\bot / f]$

\item replace every $\tscore (N)$ by $\transform{N}$ (we set aside the requirement that $\tscore$ is undefined on negative numbers)
\end{enumerate}
\lo{The above rules can be defined formally.}
\fi
\begin{align*}
\transform{\tsample} &:= r\\
\transform{y} &:= y \\
\transform{\lambda y . N} &:= \lambda y . \transform{N}\\
\transform{M_1 \, M_2} &:= \transform{M_1} \, \transform{M_2} \\
n \geq 0, \; \transform{f \, M_1 \cdots M_n} &:= (\lambda z_1 \cdots z_n . r) \, \transform{M_1} \cdots \transform{M_n} \\
\transform{\tif{B}{M_1}{M_2}} &:= \big(\lambda z . (\transform{M_1} + \transform{M_2})\big) \, \transform{B} \\
\transform{\tY \, f \, y . N} &:= \lambda y . \transform{N}[\bot / f]\\
\transform{\tscore (N)} &:= \transform{N}
\end{align*}
In the above, variables $z, z_1, \cdots, z_n$ are assumed to be fresh; $f$ ranges over primitive functions and numerals.
In transforming $\tscore (N)$, we set aside the requirement that $\tscore$ is undefined on negative numbers.

The idea is that $\transform{M}$ captures the initial, non-recursive operational behaviour of $M$, so that $\transform{M}$ simulates the reduction of $M$ until the latter reaches a value, or a term of the form $E[\tY \lambda x. N]$.

It is straightforward to see that for every trace $s \in S$, there is a reduction sequence of $\transform{M}$ that simulates (an initial subsequene of) the reduction of $M$ under $s$;
and the simulation can be made lockstep.

Finally thanks to the following
\begin{theorem}[de Groote]
\label{thm:de groote}
The simply-typed nondeterministic lambda-calculus is strongly normalising. \qed
\end{theorem}
we have:

\begin{enumerate}
\item Every $\transform{M}$-reduction terminates on reaching $r$, or a term of the shape $E[\bot N_1 \cdots N_n]$.

\item Further there is a finite bound (say $l$) on the length of such $\transform{M}$-reduction sequences; and $l$ bounds the stopping time $T_1$.
\end{enumerate}

\lo{Note: \cite{DBLP:conf/lfcs/Groote94} considered a call-by-name version of a simply-typed nondeterministic lambda-calculus, and proved it strongly normalising, using a reducibility argument \`a la Tait.
One should check that it extends to the call-by-value calculus.
}

\iffalse
@inproceedings{DBLP:conf/lfcs/Groote94,
  author    = {Philippe de Groote},
  editor    = {Anil Nerode and
               Yuri V. Matiyasevich},
  title     = {Strong Normalization in a Non-Deterministic Typed Lambda-Calculus},
  booktitle = {Logical Foundations of Computer Science, Third International Symposium,
               LFCS'94, St. Petersburg, Russia, July 11-14, 1994, Proceedings},
  series    = {Lecture Notes in Computer Science},
  volume    = {813},
  pages     = {142--152},
  publisher = {Springer},
  year      = {1994},
  url       = {https://doi.org/10.1007/3-540-58140-5\_15},
  doi       = {10.1007/3-540-58140-5\_15},
  timestamp = {Tue, 14 May 2019 10:00:54 +0200},
  biburl    = {https://dblp.org/rec/conf/lfcs/Groote94.bib},
  bibsource = {dblp computer science bibliography, https://dblp.org}
}
\fi